\documentclass{article}

\usepackage[ngerman]{babel}
\usepackage[utf8]{inputenc}
\usepackage[T1]{fontenc}
\usepackage{hyperref}
\usepackage{csquotes}

\usepackage[
    backend=biber,
    style=apa,
    sortlocale=de_DE,
    natbib=true,
    url=false,
    doi=false,
    sortcites=true,
    sorting=nyt,
    isbn=false,
    hyperref=true,
    backref=false,
    giveninits=false,
    eprint=false]{biblatex}
\addbibresource{../references/bibliography.bib}

\title{Review des Papers "Ethik im Umgang mit Daten" von Pietro D'ambrossio}
\author{Eduard Gabrielyan}
\date{\today}

\begin{document}
\maketitle

\abstract{
    Dies ist ein Review der Arbeit zum Thema Ethik im Umgang mit Daten von Pietro D'ambrossio.
}

\section{Review}
\subsection{Kapitel 1: Einleitung}
\textbf{Positive Aspekte:}
\newline
- sehr kreative Einleitung. Ein Text der von KI generiert wurde ist ein gutes Beispiel. Es gibt eine kurze Definition der Künstlichen Intelligenz und auch gleichzeitig ein Vorstellung, wie ein KI generierter Text aussieht
\newline
\newline
\textbf{Negative Aspekte:}
\newline
- gibt es hier nicht wirklich

\subsection{Kapitel 2: Risiken von Bias in KI}
\textbf{Positive Aspekte:}
\newline
- Das Kapitel hebt hervor, wie wichtig es ist, KI und Daten verantwortungsvoll zu nutzen. Das ist sehr wichtig.
\newline
- Es gibt konkrete Vorschläge, wie man Bias in KI-Systemen verringern kann, zum Beispiel durch diverse Datenquellen und transparente Algorithmen.
\newline
\newline
\textbf{Negative Aspekte:}
\newline
- Die Struktur könnte besser sein. Es gibt einige Wiederholungen und die Übergänge zwischen den Abschnitten sind nicht immer klar.

\subsection{Kapitel 3: Lösungen zur Reduzierung von Bias}

\textbf{Positive Aspekte:}
\newline
-Die vorgeschlagenen Lösungen sind praktisch und umsetzbar. Dazu gehören Bias-Training und Ethikrichtlinien.
\newline
- Die Lösungsansätze sind gut erklärt und leicht verständlich.
\newline
\newline
\textbf{Negative Aspekte:}
\newline
- ist ein sehr gutes Kapitel an sich, Ich persönlich sehe hier keine negative Aspekte

\subsection{Kapitel 4: Fazit}
\textbf{Positive Aspekte:}
\newline
- Das Fazit fasst die wichtigsten Erkenntnisse gut zusammen und betont, wie wichtig ein verantwortungsvoller Umgang mit KI und Daten ist.
\newline
- Es wird betont, wie wichtig es ist, die Rechte der Kunden zu schützen. Das ist ein sehr wichtiger Punkt.
\newline
\newline
\textbf{Negative Aspekte:}
\newline
- Das Fazit könnte detaillierter sein. Eine tiefere Reflexion oder konkrete Handlungsempfehlungen fehlen.
\subsection{Gesamteindruck}
Die Arbeit gibt eine gute Einführung in das Thema KI und deren ethischen Herausforderungen. Die Definitionen und Erklärungen sind klar und verständlich. Die Betonung auf ethische Verantwortung ist sehr wichtig. Es gibt jedoch Verbesserungspotential in der Tiefe der Analyse. Mehr Details und bessere Struktur würden die Arbeit weiter verbessern.


\printbibliography

\end{document}
