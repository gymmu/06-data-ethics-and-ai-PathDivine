\chapter{Unternehmen und KI, Daten, Ethik}
\label{chap:methode}

In diesem Kapitel werde ich auf meine Fragestellung eingehen.
\newline
\textbf{Meine Fragestellung}:
\newline
Wie können Unternehmen ethisch korrekt mit durch KI generierten Daten umgehen?

\vspace{4mm}

\includegraphics[width=1.0\textwidth]{Verantwortung.jpeg}

\newpage

\section{Verantwortung und Ethik ind er KI}
Unternehmen tragen die Verantwortung, eine sichere und transparente KI-Entwicklung zu gewährleisten, Selbstverpflichtungen einzuholen und ethische Prinzipien für den Einsatz von KI-Systemen zu definieren. Sie müssen sicherstellen, dass Kundendaten angemessen verwaltet und geschützt werden und dass Kunden darüber informiert sind, ob sie mit einem Chatbot oder einem echten Menschen sprechen.

\section{Ethischer Umgang mit Daten in Unternehmen}

Unternehmen können sicherstellen, dass sie Daten ethisch verwenden, indem sie Stakeholdern und Mitarbeitern helfen, die ethischen Fragen rund um Daten zu verstehen. Sie sollten klar erklären, wie sie mit Daten umgehen, und moderne Datenschutzlösungen einführen. Storage kann dabei unterstützen, gute Praktiken im Bereich der Datenethik zu verbessern.

\section{Informationspflichten bei KI-generierten Daten}
Unternehmen müssen ihren Kunden klar sagen, wie ihre KI die Daten sammelt, verarbeitet und mit wem diese geteilt werden. Kunden sollten wissen, ob sie mit einem Chatbot oder einem echten Menschen sprechen und welche Rechte sie bei ihren Daten haben. Es ist wichtig, klare  Regeln zu haben, um sicher zu sein, dass die KI ethisch und verantwortungsvoll genutzt wird.

\section{Datenethik}

Datenethik bezieht sich darauf, wie Daten verwaltet, verarbeitet und gespeichert werden, sowie auf moralische Fragen im Zusammenhang mit ihrer Nutzung. Es befasst sich mit der Frage, wie Organisationen Daten auf ethische Weise erfassen, speichern und nutzen können und welche Rechte der Kunden geschützt werden müssen.

\section{Risiken}
Die Risiken von KI sind, dass Unternehmen ein Gleichgewicht zwischen Schnelligkeit und Vertrauen finden müssen, wenn sie KI einführen. Ein Ansatz, der auf Risiken basiert, ist notwendig um Vorteile zu erzielen und Vertrauen aufzubauen. Die ständige Entdeckung von KI-Schwachstellen zeigt, wie anfällig KI-Tools sind. Durch den Einsatz von KI-Tools entstehen neue Sicherheitsrisiken, da KI nicht nur Angreifern hilft, sondern auch neue Arten von Risiken schafft, die beachtet werden müssen.