\subsection{Künstliche Intelligenz Training}

In einem dreistufigen Prozess findet die Ausbildung einer KI statt.
Ein Computeralgorithmus wird zunächst mit Daten versorgt, um Prognosen zu generieren und deren Präzision zu beurteilen.
Anschließend wird die Leistung des trainierten Modells an bisher ungesehenen Daten bewertet.
Beim KI-Training gibt es zwei primäre Ansätze: das überwachte Lernen, bei dem Eingabe- und Ausgabedaten gekennzeichnet sind, sowie das nicht überwachte Lernen, bei dem diese nicht vorhanden sind.
\newline
Deep Learning ist ein intensiver Prozess für den Computer, der große Mengen an Rechenleistung erfordert und auf einem hochgradig geschichteten Netzwerk aus tiefen neuronalen Pfaden basiert. Jedes Neuron des Netzwerks erzeugt komplexe Muster und Assoziationen durch mathematische Funktionen, die mit Daten gefüttert werden. Deep Learning ist eine Technik des maschinellen Lernens, die es ermöglicht, Daten zu analysieren und bessere Vorhersagen zu treffen. \citep{ai-training-cw}
\newline
Maschinelles Lernen ist ein Teilbereich der künstlichen Intelligenz, bei dem Algorithmen verwendet werden, um Beziehungen zwischen Variablen zu entdecken und aus diesen Lektionen zu lernen. Durch die Verwendung von großen Datenmengen und hoher Rechenleistung können Maschinen selbstständig eine Aufgabe erlernen und sich verbessern, ähnlich wie Kinder durch Erfahrung lernen. Im Gegensatz zu herkömmlichen Algorithmen wird beim maschinellen Lernen kein Lösungsweg modelliert, sondern die Maschine orientiert sich an einem vorgegebenen Gütekriterium und dem Informationsgehalt der Daten.