\documentclass{article}

\usepackage[ngerman]{babel}
\usepackage[utf8]{inputenc}
\usepackage[T1]{fontenc}
\usepackage{hyperref}
\usepackage{csquotes}

\usepackage[
    backend=biber,
    style=apa,
    sortlocale=de_DE,
    natbib=true,
    url=false,
    doi=false,
    sortcites=true,
    sorting=nyt,
    isbn=false,
    hyperref=true,
    backref=false,
    giveninits=false,
    eprint=false]{biblatex}
\addbibresource{../references/bibliography.bib}

\title{Notizen zum Projekt Data Ethics}
\author{Eduard Gabrielyan}
\date{\today}

\begin{document}
\maketitle

\abstract{
    Dieses Dokument ist eine Sammlung von Notizen zu dem Projekt. Die Struktur innerhalb des
    Projektes ist gleich ausgelegt wie in der Hauptarbeit, somit kann hier einfach geschrieben
    werden, und die Teile die man verwenden möchte, kann man direkt in die Hauptdatei ziehen.
    \newline
    \newline
    Meine Fragestellung zum Thema: Wie können Unternehmen ethisch korrekt mit durch KI generierten Daten umgehen?
}

\tableofcontents

\section{Künstliche Intelligenz}
\label{sec:ai}

In diesem Abschnitt sind meine Notizen zu künstlicher Intelligenz zu finden und wie man die KI traniert.
\newline
Künstliche Intelligenz ist ein Teilgebiet der Informatik und beschäftigt sich mit maschinellem Lernen \citep{ai-wikipedia}.
\newline
Künstliche Intelligenz ist wie ein kluger Computer, der durch Programme oder Lernen lernt. Besonders das maschinelle Lernen hat sich weiterentwickelt, weil es viele Daten und starke Computer gibt. Dabei lernt der Computer selbst, indem er die Daten analysiert, zum Beispiel wie Roboter, die lernen, Objekte zu greifen und zu transportieren.
\subsection{Künstliche Intelligenz Training}

In einem dreistufigen Prozess findet die Ausbildung einer KI statt.
Ein Computeralgorithmus wird zunächst mit Daten versorgt, um Prognosen zu generieren und deren Präzision zu beurteilen.
Anschließend wird die Leistung des trainierten Modells an bisher ungesehenen Daten bewertet.
Beim KI-Training gibt es zwei primäre Ansätze: das überwachte Lernen, bei dem Eingabe- und Ausgabedaten gekennzeichnet sind, sowie das nicht überwachte Lernen, bei dem diese nicht vorhanden sind. \citep{ai-training-cw}
\subsection{Künstliche Intelligenz Gefärlichkeit}

Nein, die künstliche Intelligenz (KI) ist selber nicht gefährlich. Die Gefahr liegt vielmehr in ihrem falschen Einsatz und dem Potenzial für Missbrauch. Es wird betont, dass die KI immer nützlicher werden kann, solange sie richtig eingesetzt wird und Leitplanken definiert werden. \citep{ai-wikipedia}
\subsection{Verantwortung und Ethik in der KI}

Unternehmen tragen die Verantwortung, eine sichere und transparente KI-Entwicklung zu gewährleisten, Selbstverpflichtungen einzuholen und ethische Prinzipien für den Einsatz von KI-Systemen zu definieren. Sie müssen sicherstellen, dass Kundendaten angemessen verwaltet und geschützt werden und dass Kunden darüber informiert sind, ob sie mit einem Chatbot oder einem realen Menschen interagieren.\citep{ai-res-cmm360}
\section{Daten und Ethik}
\subsection{Ethischer Umgang mit Daten in Unternehmen}

Unternehmen können einen ethischen Umgang mit Daten sicherstellen, indem sie Stakeholdern und Mitarbeitern helfen, die ethischen Überlegungen im Zusammenhang mit Daten zu verstehen, organisatorische Praktiken im Zusammenhang mit der Datennutzung klar kommunizieren und moderne Datenschutzlösungen implementieren. Storage kann dabei helfen, Best Practices im Bereich der Datenethik zu verbessern.
\citep{ai-ethik-pure}

\subsection{Informationspflichten bei KI-generierten Daten}

Unternehmen haben die Informationspflicht, ihren Kunden gegenüber transparent zu sein, wie ihre KI Daten sammelt, verarbeitet und mit wem sie geteilt werden. Kunden sollten darüber informiert werden, ob sie mit einem Chatbot oder einem echten Menschen interagieren und welche Rechte sie bezüglich ihrer Daten haben. Es ist wichtig, klare und transparente Richtlinien zu haben, um sicherzustellen, dass die KI ethisch und verantwortungsvoll eingesetzt wird.
\citep{ai-res-cmm360}

\subsection{Datenethik}

Unternehmen haben die Informationspflicht, ihren Kunden gegenüber transparent zu sein, wie ihre KI Daten sammelt, verarbeitet und mit wem sie geteilt werden. Kunden sollten darüber informiert werden, ob sie mit einem Chatbot oder einem echten Menschen interagieren und welche Rechte sie bezüglich ihrer Daten haben. Es ist wichtig, klare und transparente Richtlinien zu haben, um sicherzustellen, dass die KI ethisch und verantwortungsvoll eingesetzt wird.
\citep{ai-ethik-pure}
\section{Textvorbereitung für das Enddokument}
\subsection{Einleitung}

Künstliche Intelligenz ist ein Teil der Informatik und beschäftigt sich mit maschinellem Lernen.
Künstliche Intelligenz ist wie ein schlauer Computer, der durch Programme oder Lernen klüger wird. Maschinelles Lernen ist sehr wichtig geworden, weil es viele Daten und starke Computer gibt. Dabei lernt der Computer selbst, indem er die Daten anschaut, wie Roboter, die lernen, Dinge zu greifen und zu bewegen.

Die Ausbildung einer KI passiert in drei Schritten. Erst bekommt ein Computerprogramm Daten, um Vorhersagen zu machen und deren Genauigkeit zu überprüfen. Danach wird die Leistung des Programms mit neuen Daten getestet. Es gibt zwei Hauptmethoden beim KI-Training: Beim überwachten Lernen gibt es markierte Eingabe- und Ausgabedaten, beim nicht überwachten Lernen gibt es diese Markierungen nicht.

\subsection{Fazit}

Das Kapitel behandelt, wie Unternehmen ethisch korrekt mit durch KI generierten Daten umgehen können. Es betont die Verantwortung der Unternehmen für eine sichere und transparente KI-Entwicklung sowie den Schutz und die angemessene Verwaltung von Kundendaten. 
\newline
\newline
Unternehmen sollen Stakeholder und Mitarbeiter über ethische Datenfragen aufklären, moderne Datenschutzlösungen einführen und klare Kommunikationsrichtlinien zum Umgang mit Daten etablieren. Kunden müssen über die Datenverarbeitung durch KI informiert werden und ihre Rechte kennen, insbesondere wenn sie mit Chatbots interagieren.
\newline
Die Datenethik konzentriert sich auf die moralischen Aspekte der Datennutzung, während Unternehmen sicherstellen müssen, dass Kundenrechte gewahrt bleiben. Zudem wird auf die Risiken der KI hingewiesen: Unternehmen müssen ein Gleichgewicht zwischen Geschwindigkeit und Vertrauen finden und neue Sicherheitsrisiken beachten.
\newline
\newline
Zusammengefasst, ist ein verantwortungsvoller und ethischer Umgang mit KI und den daraus resultierenden Daten essenziell für das Vertrauen der Kunden und die Datensicherheit.

\printbibliography

\end{document}
