\documentclass{article}

\usepackage[ngerman]{babel}
\usepackage[utf8]{inputenc}
\usepackage[T1]{fontenc}
\usepackage{hyperref}
\usepackage{csquotes}

\usepackage[
    backend=biber,
    style=apa,
    sortlocale=de_DE,
    natbib=true,
    url=false,
    doi=false,
    sortcites=true,
    sorting=nyt,
    isbn=false,
    hyperref=true,
    backref=false,
    giveninits=false,
    eprint=false]{biblatex}
\addbibresource{../references/bibliography.bib}

\title{Notizen zum Projekt Data Ethics}
\author{Eduard Gabrielyan}
\date{\today}

\begin{document}
\maketitle

\abstract{
    Dieses Dokument ist eine Sammlung von Notizen zu dem Projekt. Die Struktur innerhalb des
    Projektes ist gleich ausgelegt wie in der Hauptarbeit, somit kann hier einfach geschrieben
    werden, und die Teile die man verwenden möchte, kann man direkt in die Hauptdatei ziehen.
    \newline
    \newline
    Meine Fragestellung zum Thema: Wie können Unternehmen ethisch korrekt mit durch KI generierten Daten umgehen?
}

\tableofcontents

\section{Künstliche Intelligenz}
\label{sec:ai}

In diesem Abschnitt sind meine Notizen zu künstlicher Intelligenz zu finden und wie man die KI traniert.
\newline
Künstliche Intelligenz ist ein Teilgebiet der Informatik und beschäftigt sich mit maschinellem Lernen \citep{ai-wikipedia}.
\newline
Künstliche Intelligenz ist wie ein kluger Computer, der durch Programme oder Lernen lernt. Besonders das maschinelle Lernen hat sich weiterentwickelt, weil es viele Daten und starke Computer gibt. Dabei lernt der Computer selbst, indem er die Daten analysiert, zum Beispiel wie Roboter, die lernen, Objekte zu greifen und zu transportieren.
\subsection{Künstliche Intelligenz Training}

In einem dreistufigen Prozess findet die Ausbildung einer KI statt.
Ein Computeralgorithmus wird zunächst mit Daten versorgt, um Prognosen zu generieren und deren Präzision zu beurteilen.
Anschließend wird die Leistung des trainierten Modells an bisher ungesehenen Daten bewertet.
Beim KI-Training gibt es zwei primäre Ansätze: das überwachte Lernen, bei dem Eingabe- und Ausgabedaten gekennzeichnet sind, sowie das nicht überwachte Lernen, bei dem diese nicht vorhanden sind. \citep{ai-training-cw}

\printbibliography

\end{document}
