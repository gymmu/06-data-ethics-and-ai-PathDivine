\section{Daten und Ethik}
\subsection{Ethischer Umgang mit Daten in Unternehmen}

Unternehmen können einen ethischen Umgang mit Daten sicherstellen, indem sie Stakeholdern und Mitarbeitern helfen, die ethischen Überlegungen im Zusammenhang mit Daten zu verstehen, organisatorische Praktiken im Zusammenhang mit der Datennutzung klar kommunizieren und moderne Datenschutzlösungen implementieren. Storage kann dabei helfen, Best Practices im Bereich der Datenethik zu verbessern.
\citep{ai-ethik-pure}

\subsection{Informationspflichten bei KI-generierten Daten}

Unternehmen haben die Informationspflicht, ihren Kunden gegenüber transparent zu sein, wie ihre KI Daten sammelt, verarbeitet und mit wem sie geteilt werden. Kunden sollten darüber informiert werden, ob sie mit einem Chatbot oder einem echten Menschen interagieren und welche Rechte sie bezüglich ihrer Daten haben. Es ist wichtig, klare und transparente Richtlinien zu haben, um sicherzustellen, dass die KI ethisch und verantwortungsvoll eingesetzt wird.
\citep{ai-res-cmm360}

\subsection{Datenethik}

Datenethik bezieht sich darauf, wie Daten verwaltet, verarbeitet und gespeichert werden, sowie auf moralische Fragen im Zusammenhang mit ihrer Nutzung. Es befasst sich mit der Frage, wie Organisationen Daten auf ethische Weise erfassen, speichern und nutzen können und welche Kundenrechte geschützt werden müssen.
\citep{ai-ethik-pure}