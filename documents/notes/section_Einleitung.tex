\section{Textvorbereitung für das Enddokument}
\subsection{Einleitung}

Künstliche Intelligenz ist ein Teil der Informatik und beschäftigt sich mit maschinellem Lernen.
Künstliche Intelligenz ist wie ein schlauer Computer, der durch Programme oder Lernen klüger wird. Maschinelles Lernen ist sehr wichtig geworden, weil es viele Daten und starke Computer gibt. Dabei lernt der Computer selbst, indem er die Daten anschaut, wie Roboter, die lernen, Dinge zu greifen und zu bewegen.

Die Ausbildung einer KI passiert in drei Schritten. Erst bekommt ein Computerprogramm Daten, um Vorhersagen zu machen und deren Genauigkeit zu überprüfen. Danach wird die Leistung des Programms mit neuen Daten getestet. Es gibt zwei Hauptmethoden beim KI-Training: Beim überwachten Lernen gibt es markierte Eingabe- und Ausgabedaten, beim nicht überwachten Lernen gibt es diese Markierungen nicht.

\subsection{Fazit}

Das Kapitel behandelt, wie Unternehmen ethisch korrekt mit durch KI generierten Daten umgehen können. Es betont die Verantwortung der Unternehmen für eine sichere und transparente KI-Entwicklung sowie den Schutz und die angemessene Verwaltung von Kundendaten. 
\newline
\newline
Unternehmen sollen Stakeholder und Mitarbeiter über ethische Datenfragen aufklären, moderne Datenschutzlösungen einführen und klare Kommunikationsrichtlinien zum Umgang mit Daten etablieren. Kunden müssen über die Datenverarbeitung durch KI informiert werden und ihre Rechte kennen, insbesondere wenn sie mit Chatbots interagieren.
\newline
Die Datenethik konzentriert sich auf die moralischen Aspekte der Datennutzung, während Unternehmen sicherstellen müssen, dass Kundenrechte gewahrt bleiben. Zudem wird auf die Risiken der KI hingewiesen: Unternehmen müssen ein Gleichgewicht zwischen Geschwindigkeit und Vertrauen finden und neue Sicherheitsrisiken beachten.
\newline
\newline
Zusammengefasst, ist ein verantwortungsvoller und ethischer Umgang mit KI und den daraus resultierenden Daten essenziell für das Vertrauen der Kunden und die Datensicherheit.