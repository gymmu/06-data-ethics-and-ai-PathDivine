\section{Textvorbereitung für das Enddokument}
\subsection{Einleitung}

Künstliche Intelligenz ist ein Teil der Informatik und beschäftigt sich mit maschinellem Lernen.
Künstliche Intelligenz ist wie ein schlauer Computer, der durch Programme oder Lernen klüger wird. Maschinelles Lernen ist sehr wichtig geworden, weil es viele Daten und starke Computer gibt. Dabei lernt der Computer selbst, indem er die Daten anschaut, wie Roboter, die lernen, Dinge zu greifen und zu bewegen.

Die Ausbildung einer KI passiert in drei Schritten. Erst bekommt ein Computerprogramm Daten, um Vorhersagen zu machen und deren Genauigkeit zu überprüfen. Danach wird die Leistung des Programms mit neuen Daten getestet. Es gibt zwei Hauptmethoden beim KI-Training: Beim überwachten Lernen gibt es markierte Eingabe- und Ausgabedaten, beim nicht überwachten Lernen gibt es diese Markierungen nicht.